%\renewcommand{\baselinestretch}{2.0}   %---% line spacing, double
%\renewcommand{\baselinestretch}{1.5}   %---% line spacing, one + half
\renewcommand{\baselinestretch}{1.0}   %---% line spacing, single

%  -  a percentage sign comments out lines in laTeX.

%  - these are useful in a thesis.
%  - equations are numbered from 1 in each section.
\renewcommand{\thesection}{\arabic{section}}
\renewcommand{\thesubsection}{\arabic{section}.\arabic{subsection}}
%\renewcommand{\theequation}{\thesection.\arabic{equation}}
%\renewcommand{\thefigure}{\thesection.\arabic{figure}}
\renewcommand{\thefigure}{\arabic{figure}}
%\renewcommand{\thetable}{\thesection.\arabic{table}}
\renewcommand{\thefootnote}{\roman{footnote}}
%\newcounter{appendix}
%\renewcommand{\theappendix}{\Alph{appendix}}

%\renewcommand{\thepage}{\thesection.\arabic{page}}
%\newcommand{\resetcounters}{\setcounter{equation}{0} \addtocounter{section}{1}}
\newcommand{\resetcounters}{\setcounter{equation}{0} }

%  - these change the set-up of the page
\oddsidemargin = 0.0mm
\textwidth     = 165.0mm
\topmargin     = -10.0mm
\textheight    = 230.0mm
 \parindent=0 cm
%  - commands can be set up by the user.
%
% LaTeX user's definition file
%
% put equation numbers in ( )'s.
\newcommand{\refeqn}[1]{{\rm (\ref{#1})}}
\newcommand{\np}{\hspace*{1cm}}
\newcommand{\shellr     }{ {\unitlength = 1 pt
                           \begin{picture}(8,7.4)
                           \put(0,0){R}
                           \put(0.4,0.3){\line(0,1){6.1}}
                           \put(0.4,0.3){\line(-1,0){1}}
                           \put(0.4,6.4){\line(-1,0){1}}
                           \end{picture}  }  }

\newenvironment{aside}
{{\underline{\bf Aside :}} \it \begin{quote}}{\end{quote}}

\newenvironment{caveat}
{{\underline{\bf Caveat :}} \it \begin{quote}}{\end{quote}}

%% QJ style appendices (pinched and tweaked form of bits from qj.sty):

\def\appsection#1{
 % first the admin:
  \refstepcounter{section} % increments section but maintains correct
                           % labelling and also resets sub-counters 
                           % (ie subsection etc)
  \addcontentsline{toc}{section}{\protect\numberline{\thesection} #1}
                           % does all the correct table of contents
                           % entry. Taken from general section stuff in
                           % /opt/local/lib/tex/TeX3.14/inputs/latex.tex
 % now the actual section header:
  \vskip 1.5\baselineskip 
  \noindent\parbox{\hsize}{\centering\sc APPENDIX \thesection}
  \vskip 0.75\baselineskip
  \noindent\parbox{\hsize}{\centering \bf #1}
  \vskip 0.5\baselineskip 
}

%% mods to appendix command

\def\appendix{
  \par  
 \setcounter{section}{0}                    % start section numbering again
 \renewcommand{\thesection}{\Alph{section}} % reference section as alphabetic 
  \renewcommand{\thesubsection}{\Alph{section}.\arabic{subsection}}
%  \renewcommand{\section}{\appsection}       % redefine what a section header                                              % looks like
\renewcommand{\theequation}{\thesection.\arabic{equation}}
}

\def\ltapp{\raisebox{-.4ex}{$\ \stackrel{<}{{\scriptstyle \sim}} \ $}}  
\def\gtapp{\raisebox{-.4ex}{$\ \stackrel{>}{{\scriptstyle \sim}} \ $}}

%%%%%%%%%%%%%%%%%%%%%%%%%%%%%%%%%%%%%%%%%%%%%%%%%%%%%%%%%%%%%%%%%%%%%%%%%%%%%%%
%% Include modified fancyhdr.sty to put date in top right corner

% fancyhdr.sty version 1.99b %% Modified by Nigel Wood 13/4/00 to put
%                            %% date in header
% Fancy headers and footers.
% Piet van Oostrum, Dept of Computer Science, University of Utrecht
% Padualaan 14, P.O. Box 80.089, 3508 TB Utrecht, The Netherlands
% Telephone: +31 30 2531806. Email: piet@cs.ruu.nl
% Sep 16, 1994
% version 1.4: Correction for use with \reversemargin
% Sep 29, 1994:
% version 1.5: Added the \iftopfloat, \ifbotfloat and \iffloatpage commands
% Oct 4, 1994:
% version 1.6: Reset single spacing in headers/footers for use with
% setspace.sty or doublespace.sty
% Oct 4, 1994:
% version 1.7: changed \let\@mkboth\markboth to
% \def\@mkboth{\protect\markboth} to make it more robust
% Dec 5, 1994:
% version 1.8: corrections for amsbook/amsart: define \@chapapp and (more
% importantly) use the \chapter/sectionmark definitions from ps@headings if
% they exist (which should be true for all standard classes).
% May 31, 1995:
% version 1.9: The proposed \renewcommand{\headrulewidth}{\iffloatpage...
% construction in the doc did not work properly with the fancyplain style. 
% June 1, 1995:
% version 1.91: The definition of \@mkboth wasn't restored on subsequent
% \pagestyle{fancy}'s.
% June 1, 1995:
% version 1.92: The sequence \pagestyle{fancyplain} \pagestyle{plain}
% \pagestyle{fancy} would erroneously select the plain version.
% June 1, 1995:
% version 1.93: \fancypagestyle command added.
% Dec 11, 1995:
% version 1.94: suggested by Conrad Hughes <chughes@maths.tcd.ie>
% CJCH, Dec 11, 1995: added \footruleskip to allow control over footrule
% position (old hardcoded value of .3\normalbaselineskip is far too high
% when used with very small footer fonts).
% Jan 31, 1996:
% version 1.95: call \@normalsize in the reset code if that is defined,
% otherwise \normalsize.
% this is to solve a problem with ucthesis.cls, as this doesn't
% define \@currsize. Unfortunately for latex209 calling \normalsize doesn't
% work as this is optimized to do very little, so there \@normalsize should
% be called. Hopefully this code works for all versions of LaTeX known to
% mankind.  
% April 25, 1996:
% version 1.96: initialize \headwidth to a magic (negative) value to catch
% most common cases that people change it before calling \pagestyle{fancy}.
% Note it can't be initialized when reading in this file, because
% \textwidth could be changed afterwards. This is quite probable.
% We also switch to \MakeUppercase rather than \uppercase and introduce a
% \nouppercase command for use in headers. and footers.
% May 3, 1996:
% version 1.97: Two changes:
% 1. Undo the change in version 1.8 (using the pagestyle{headings} defaults
% for the chapter and section marks. The current version of amsbook and
% amsart classes don't seem to need them anymore. Moreover the standard
% latex classes don't use \markboth if twoside isn't selected, and this is
% confusing as \leftmark doesn't work as expected.
% 2. include a call to \ps@empty in ps@@fancy. This is to solve a problem
% in the amsbook and amsart classes, that make global changes to \topskip,
% which are reset in \ps@empty. Hopefully this doesn't break other things.
% May 7, 1996:
% version 1.98:
% Added % after the line  \def\nouppercase
% May 7, 1996:
% version 1.99: This is the alpha version of fancyhdr 2.0
% Introduced the new commands \fancyhead, \fancyfoot, and \fancyhf.
% Changed \headrulewidth, \footrulewidth, \footruleskip to
% macros rather than length parameters, In this way they can be
% conditionalized and they don't consume length registers. There is no need
% to have them as length registers unless you want to do calculations with
% them, which is unlikely. Note that this may make some uses of them
% incompatible (i.e. if you have a file that uses \setlength or \xxxx=
% May 10, 1996:
% version 1.99a:
% Added a few more % signs
% May 10, 1996:
% version 1.99b:
% Changed the syntax of \f@nfor to be resistent to catcode changes of :=
% Removed the [1] from the defs of \lhead etc. because the parameter is
% consumed by the \@[xy]lhead etc. macros.

\let\fancy@def\gdef

\def\if@mpty#1#2#3{\def\temp@ty{#1}\ifx\@empty\temp@ty #2\else#3\fi}

% Usage: \@forc \var{charstring}{command to be executed for each char}
% This is similar to LaTeX's \@tfor, but expands the charstring.

\def\@forc#1#2#3{\expandafter\f@rc\expandafter#1\expandafter{#2}{#3}}
\def\f@rc#1#2#3{\def\temp@ty{#2}\ifx\@empty\temp@ty\else
                                    \f@@rc#1#2\f@@rc{#3}\fi}
\def\f@@rc#1#2#3\f@@rc#4{\def#1{#2}#4\f@rc#1{#3}{#4}}

% Usage: \f@nfor\name:=list\do{body}
% Like LaTeX's \@for but an empty list is treated as a list with an empty
% element

\newcommand{\f@nfor}[3]{\edef\@fortmp{#2}%
    \expandafter\@forloop#2,\@nil,\@nil\@@#1{#3}}

% Usage: \def@ult \cs{defaults}{argument}
% sets \cs to the characters from defaults appearing in argument
% or defaults if it would be empty. All characters are lowercased.

\newcommand\def@ult[3]{%
    \edef\temp@a{\lowercase{\edef\noexpand\temp@a{#3}}}\temp@a
    \def#1{}%
    \@forc\tmpf@ra{#2}%
        {\expandafter\if@in\tmpf@ra\temp@a{\edef#1{#1\tmpf@ra}}{}}%
    \ifx\@empty#1\def#1{#2}\fi}
% 
% \if@in <char><set><truecase><falsecase>
%
\newcommand{\if@in}[4]{%
    \edef\temp@a{#2}\def\temp@b##1#1##2\temp@b{\def\temp@b{##1}}%
    \expandafter\temp@b#2#1\temp@b\ifx\temp@a\temp@b #4\else #3\fi}

\newcommand{\fancyhead}{\@ifnextchar[{\f@ncyhf h}{\f@ncyhf h[]}}
\newcommand{\fancyfoot}{\@ifnextchar[{\f@ncyhf f}{\f@ncyhf f[]}}
\newcommand{\fancyhf}{\@ifnextchar[{\f@ncyhf {}}{\f@ncyhf {}[]}}

% The header and footer fields are stored in command sequences with
% names of the form: \f@ncy<x><y><z> with <x> for [eo], <y> form [lcr]
% and <z> from [hf].

\def\f@ncyhf#1[#2]#3{%
    \def\temp@c{}%
    \@forc\tmpf@ra{#2}%
        {\expandafter\if@in\tmpf@ra{eolcrhf,EOLCRHF}%
            {}{\edef\temp@c{\temp@c\tmpf@ra}}}%
    \ifx\@empty\temp@c\else
        \ifx\PackageError\undefined
        \errmessage{Illegal char `\temp@c' in fancyhdr argument:
          [#2]}\else
        \PackageError{Fancyhdr}{Illegal char `\temp@c' in fancyhdr argument:
          [#2]}{}\fi
    \fi
    \f@nfor\temp@c{#2}%
        {\def@ult\f@@@eo{eo}\temp@c
         \def@ult\f@@@lcr{lcr}\temp@c
         \def@ult\f@@@hf{hf}{#1\temp@c}%
         \@forc\f@@eo\f@@@eo
             {\@forc\f@@lcr\f@@@lcr
                 {\@forc\f@@hf\f@@@hf
                     {\expandafter\fancy@def\csname
                      f@ncy\f@@eo\f@@lcr\f@@hf\endcsname
                      {#3}}}}}}

% Fancyheadings version 1 commands. These are more or less depracated,
% but they continue to work.

\newcommand{\lhead}{\@ifnextchar[{\@xlhead}{\@ylhead}}
\def\@xlhead[#1]#2{\fancy@def\f@ncyelh{#1}\fancy@def\f@ncyolh{#2}}
\def\@ylhead#1{\fancy@def\f@ncyelh{#1}\fancy@def\f@ncyolh{#1}}

\newcommand{\chead}{\@ifnextchar[{\@xchead}{\@ychead}}
\def\@xchead[#1]#2{\fancy@def\f@ncyech{#1}\fancy@def\f@ncyoch{#2}}
\def\@ychead#1{\fancy@def\f@ncyech{#1}\fancy@def\f@ncyoch{#1}}

\newcommand{\rhead}{\@ifnextchar[{\@xrhead}{\@yrhead}}
\def\@xrhead[#1]#2{\fancy@def\f@ncyerh{#1}\fancy@def\f@ncyorh{#2}}
\def\@yrhead#1{\fancy@def\f@ncyerh{#1}\fancy@def\f@ncyorh{#1}}

\newcommand{\lfoot}{\@ifnextchar[{\@xlfoot}{\@ylfoot}}
\def\@xlfoot[#1]#2{\fancy@def\f@ncyelf{#1}\fancy@def\f@ncyolf{#2}}
\def\@ylfoot#1{\fancy@def\f@ncyelf{#1}\fancy@def\f@ncyolf{#1}}

\newcommand{\cfoot}{\@ifnextchar[{\@xcfoot}{\@ycfoot}}
\def\@xcfoot[#1]#2{\fancy@def\f@ncyecf{#1}\fancy@def\f@ncyocf{#2}}
\def\@ycfoot#1{\fancy@def\f@ncyecf{#1}\fancy@def\f@ncyocf{#1}}

\newcommand{\rfoot}{\@ifnextchar[{\@xrfoot}{\@yrfoot}}
\def\@xrfoot[#1]#2{\fancy@def\f@ncyerf{#1}\fancy@def\f@ncyorf{#2}}
\def\@yrfoot#1{\fancy@def\f@ncyerf{#1}\fancy@def\f@ncyorf{#1}}

\newdimen\headwidth
\newcommand{\headrulewidth}{0.4pt}
\newcommand{\footrulewidth}{\z@skip}
\newcommand{\footruleskip}{.3\normalbaselineskip}

% Fancyplain stuff shouldn't be used anymore (rather
% \fancypagestyle{plain} should be used), but it must be present for
% compatibility reasons.

\newcommand{\plainheadrulewidth}{\z@skip}
\newcommand{\plainfootrulewidth}{\z@skip}
\newif\if@fancyplain \@fancyplainfalse
\def\fancyplain#1#2{\if@fancyplain#1\else#2\fi}

\headwidth=-123456789sp %magic constant

% Command to reset various things in the headers:
% a.o.  single spacing (taken from setspace.sty)
% and the catcode of ^^M (so that epsf files in the header work if a
% verbatim crosses a page boundary)
% It also defines a \nouppercase command that disables \uppercase and
% \Makeuppercase. It can only be used in the headers and footers.
\def\fancy@reset{\restorecr
 \def\baselinestretch{1}%
 \def\nouppercase##1{{\let\uppercase\relax\let\MakeUppercase\relax##1}}%
 \ifx\undefined\@newbaseline% NFSS not present; 2.09 or 2e
   \ifx\@normalsize\undefined \normalsize % for ucthesis.cls
   \else \@normalsize \fi
 \else% NFSS (2.09) present
  \@newbaseline%
 \fi}

% Initialization of the head and foot text.

% The default values still contain \fancyplain for compatibility.
\fancyhf{} % clear all
% lefthead empty on ``plain'' pages, \rightmark on even, \leftmark on odd pages
% evenhead empty on ``plain'' pages, \leftmark on even, \rightmark on odd pages
%% Nigel Wood 13/4/00 
%% \fancyhead[el,or]{\fancyplain{}{\sl\rightmark}}
\fancyhead[el,or]{\fancyplain{}{\sl\today}}
%% Nigel Wood 13/4/00 
%% \fancyhead[er,ol]{\fancyplain{}{\sl\leftmark}}
\fancyhead[er,ol]{\fancyplain{}{}}
\fancyfoot[c]{\rm\thepage} % page number

% Put together a header or footer given the left, center and
% right text, fillers at left and right and a rule.
% The \lap commands put the text into an hbox of zero size,
% so overlapping text does not generate an errormessage.

\def\@fancyhead#1#2#3#4#5{#1\hbox to\headwidth{\fancy@reset\vbox{\hbox
{\rlap{\parbox[b]{\headwidth}{\raggedright#2\strut}}\hfill
\parbox[b]{\headwidth}{\centering#3\strut}\hfill
\llap{\parbox[b]{\headwidth}{\raggedleft#4\strut}}}\headrule}}#5}

\def\@fancyfoot#1#2#3#4#5{#1\hbox to\headwidth{\fancy@reset\vbox{\footrule
\hbox{\rlap{\parbox[t]{\headwidth}{\raggedright#2\strut}}\hfill
\parbox[t]{\headwidth}{\centering#3\strut}\hfill
\llap{\parbox[t]{\headwidth}{\raggedleft#4\strut}}}}}#5}

\def\headrule{{\if@fancyplain\let\headrulewidth\plainheadrulewidth\fi
\hrule\@height\headrulewidth\@width\headwidth \vskip-\headrulewidth}}

\def\footrule{{\if@fancyplain\let\footrulewidth\plainfootrulewidth\fi
\vskip-\footruleskip\vskip-\footrulewidth
\hrule\@width\headwidth\@height\footrulewidth\vskip\footruleskip}}

\def\ps@fancy{%
\@ifundefined{@chapapp}{\let\@chapapp\chaptername}{}%for amsbook
%
% Define \MakeUppercase for old LaTeXen.
% Note: we used \def rather than \let, so that \let\uppercase\relax (from
% the version 1 documentation) will still work.
%
%% Nigel Wood 13/4/00 
%%\@ifundefined{MakeUppercase}{\def\MakeUppercase{\uppercase}}{}%
%%\@ifundefined{chapter}{\def\sectionmark##1{\markboth
%%{\MakeUppercase{\ifnum \c@secnumdepth>\z@
%% \thesection\hskip 1em\relax \fi ##1}}{}}%
%%\def\subsectionmark##1{\markright {\ifnum \c@secnumdepth >\@ne
%% \thesubsection\hskip 1em\relax \fi ##1}}}%
%%{\def\chaptermark##1{\markboth {\MakeUppercase{\ifnum \c@secnumdepth>\m@ne
%% \@chapapp\ \thechapter. \ \fi ##1}}{}}%
%%\def\sectionmark##1{\markright{\MakeUppercase{\ifnum \c@secnumdepth >\z@
%% \thesection. \ \fi ##1}}}}%
%\csname ps@headings\endcsname % use \ps@headings defaults if they exist
\ps@@fancy
\gdef\ps@fancy{\@fancyplainfalse\ps@@fancy}%
% Initialize \headwidth if the user didn't
%
\ifdim\headwidth<0sp
%
% This catches the case that \headwidth hasn't been initialized and the
% case that the user added something to \headwidth in the expectation that
% it was initialized to \textwidth. We compensate this now. This loses if
% the user intended to multiply it by a factor. But that case is more
% likely done by saying something like \headwidth=1.2\textwidth. 
% The doc says you have to change \headwidth after the first call to
% \pagestyle{fancy}. This code is just to catch the most common cases were
% that requirement is violated.
%
    \advance\headwidth123456789sp\advance\headwidth\textwidth
\fi}
\def\ps@fancyplain{\ps@fancy \let\ps@plain\ps@plain@fancy}
\def\ps@plain@fancy{\@fancyplaintrue\ps@@fancy}
\def\ps@@fancy{%
\ps@empty % This is for amsbook/amsart, which do strange things with \topskip
\def\@mkboth{\protect\markboth}%
\def\@oddhead{\@fancyhead\@lodd\f@ncyolh\f@ncyoch\f@ncyorh\@rodd}%
\def\@oddfoot{\@fancyfoot\@lodd\f@ncyolf\f@ncyocf\f@ncyorf\@rodd}%
\def\@evenhead{\@fancyhead\@rodd\f@ncyelh\f@ncyech\f@ncyerh\@lodd}%
\def\@evenfoot{\@fancyfoot\@rodd\f@ncyelf\f@ncyecf\f@ncyerf\@lodd}%
}
\def\@lodd{\if@reversemargin\hss\else\relax\fi}
\def\@rodd{\if@reversemargin\relax\else\hss\fi}

\let\latex@makecol\@makecol
\def\@makecol{\let\topfloat\@toplist\let\botfloat\@botlist\latex@makecol}
\def\iftopfloat#1#2{\ifx\topfloat\empty #2\else #1\fi}
\def\ifbotfloat#1#2{\ifx\botfloat\empty #2\else #1\fi}
\def\iffloatpage#1#2{\if@fcolmade #1\else #2\fi}

\newcommand{\fancypagestyle}[2]{%
  \@namedef{ps@#1}{\let\fancy@def\def#2\relax\ps@fancy}}

